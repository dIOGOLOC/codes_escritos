
% Default to the notebook output style

    


% Inherit from the specified cell style.




    
\documentclass[11pt]{article}

    
    
    \usepackage[T1]{fontenc}
    % Nicer default font (+ math font) than Computer Modern for most use cases
    \usepackage{mathpazo}

    % Basic figure setup, for now with no caption control since it's done
    % automatically by Pandoc (which extracts ![](path) syntax from Markdown).
    \usepackage{graphicx}
    % We will generate all images so they have a width \maxwidth. This means
    % that they will get their normal width if they fit onto the page, but
    % are scaled down if they would overflow the margins.
    \makeatletter
    \def\maxwidth{\ifdim\Gin@nat@width>\linewidth\linewidth
    \else\Gin@nat@width\fi}
    \makeatother
    \let\Oldincludegraphics\includegraphics
    % Set max figure width to be 80% of text width, for now hardcoded.
    \renewcommand{\includegraphics}[1]{\Oldincludegraphics[width=.8\maxwidth]{#1}}
    % Ensure that by default, figures have no caption (until we provide a
    % proper Figure object with a Caption API and a way to capture that
    % in the conversion process - todo).
    \usepackage{caption}
    \DeclareCaptionLabelFormat{nolabel}{}
    \captionsetup{labelformat=nolabel}

    \usepackage{adjustbox} % Used to constrain images to a maximum size 
    \usepackage{xcolor} % Allow colors to be defined
    \usepackage{enumerate} % Needed for markdown enumerations to work
    \usepackage{geometry} % Used to adjust the document margins
    \usepackage{amsmath} % Equations
    \usepackage{amssymb} % Equations
    \usepackage{textcomp} % defines textquotesingle
    % Hack from http://tex.stackexchange.com/a/47451/13684:
    \AtBeginDocument{%
        \def\PYZsq{\textquotesingle}% Upright quotes in Pygmentized code
    }
    \usepackage{upquote} % Upright quotes for verbatim code
    \usepackage{eurosym} % defines \euro
    \usepackage[mathletters]{ucs} % Extended unicode (utf-8) support
    \usepackage[utf8x]{inputenc} % Allow utf-8 characters in the tex document
    \usepackage{fancyvrb} % verbatim replacement that allows latex
    \usepackage{grffile} % extends the file name processing of package graphics 
                         % to support a larger range 
    % The hyperref package gives us a pdf with properly built
    % internal navigation ('pdf bookmarks' for the table of contents,
    % internal cross-reference links, web links for URLs, etc.)
    \usepackage{hyperref}
    \usepackage{longtable} % longtable support required by pandoc >1.10
    \usepackage{booktabs}  % table support for pandoc > 1.12.2
    \usepackage[inline]{enumitem} % IRkernel/repr support (it uses the enumerate* environment)
    \usepackage[normalem]{ulem} % ulem is needed to support strikethroughs (\sout)
                                % normalem makes italics be italics, not underlines
    

    
    
    % Colors for the hyperref package
    \definecolor{urlcolor}{rgb}{0,.145,.698}
    \definecolor{linkcolor}{rgb}{.71,0.21,0.01}
    \definecolor{citecolor}{rgb}{.12,.54,.11}

    % ANSI colors
    \definecolor{ansi-black}{HTML}{3E424D}
    \definecolor{ansi-black-intense}{HTML}{282C36}
    \definecolor{ansi-red}{HTML}{E75C58}
    \definecolor{ansi-red-intense}{HTML}{B22B31}
    \definecolor{ansi-green}{HTML}{00A250}
    \definecolor{ansi-green-intense}{HTML}{007427}
    \definecolor{ansi-yellow}{HTML}{DDB62B}
    \definecolor{ansi-yellow-intense}{HTML}{B27D12}
    \definecolor{ansi-blue}{HTML}{208FFB}
    \definecolor{ansi-blue-intense}{HTML}{0065CA}
    \definecolor{ansi-magenta}{HTML}{D160C4}
    \definecolor{ansi-magenta-intense}{HTML}{A03196}
    \definecolor{ansi-cyan}{HTML}{60C6C8}
    \definecolor{ansi-cyan-intense}{HTML}{258F8F}
    \definecolor{ansi-white}{HTML}{C5C1B4}
    \definecolor{ansi-white-intense}{HTML}{A1A6B2}

    % commands and environments needed by pandoc snippets
    % extracted from the output of `pandoc -s`
    \providecommand{\tightlist}{%
      \setlength{\itemsep}{0pt}\setlength{\parskip}{0pt}}
    \DefineVerbatimEnvironment{Highlighting}{Verbatim}{commandchars=\\\{\}}
    % Add ',fontsize=\small' for more characters per line
    \newenvironment{Shaded}{}{}
    \newcommand{\KeywordTok}[1]{\textcolor[rgb]{0.00,0.44,0.13}{\textbf{{#1}}}}
    \newcommand{\DataTypeTok}[1]{\textcolor[rgb]{0.56,0.13,0.00}{{#1}}}
    \newcommand{\DecValTok}[1]{\textcolor[rgb]{0.25,0.63,0.44}{{#1}}}
    \newcommand{\BaseNTok}[1]{\textcolor[rgb]{0.25,0.63,0.44}{{#1}}}
    \newcommand{\FloatTok}[1]{\textcolor[rgb]{0.25,0.63,0.44}{{#1}}}
    \newcommand{\CharTok}[1]{\textcolor[rgb]{0.25,0.44,0.63}{{#1}}}
    \newcommand{\StringTok}[1]{\textcolor[rgb]{0.25,0.44,0.63}{{#1}}}
    \newcommand{\CommentTok}[1]{\textcolor[rgb]{0.38,0.63,0.69}{\textit{{#1}}}}
    \newcommand{\OtherTok}[1]{\textcolor[rgb]{0.00,0.44,0.13}{{#1}}}
    \newcommand{\AlertTok}[1]{\textcolor[rgb]{1.00,0.00,0.00}{\textbf{{#1}}}}
    \newcommand{\FunctionTok}[1]{\textcolor[rgb]{0.02,0.16,0.49}{{#1}}}
    \newcommand{\RegionMarkerTok}[1]{{#1}}
    \newcommand{\ErrorTok}[1]{\textcolor[rgb]{1.00,0.00,0.00}{\textbf{{#1}}}}
    \newcommand{\NormalTok}[1]{{#1}}
    
    % Additional commands for more recent versions of Pandoc
    \newcommand{\ConstantTok}[1]{\textcolor[rgb]{0.53,0.00,0.00}{{#1}}}
    \newcommand{\SpecialCharTok}[1]{\textcolor[rgb]{0.25,0.44,0.63}{{#1}}}
    \newcommand{\VerbatimStringTok}[1]{\textcolor[rgb]{0.25,0.44,0.63}{{#1}}}
    \newcommand{\SpecialStringTok}[1]{\textcolor[rgb]{0.73,0.40,0.53}{{#1}}}
    \newcommand{\ImportTok}[1]{{#1}}
    \newcommand{\DocumentationTok}[1]{\textcolor[rgb]{0.73,0.13,0.13}{\textit{{#1}}}}
    \newcommand{\AnnotationTok}[1]{\textcolor[rgb]{0.38,0.63,0.69}{\textbf{\textit{{#1}}}}}
    \newcommand{\CommentVarTok}[1]{\textcolor[rgb]{0.38,0.63,0.69}{\textbf{\textit{{#1}}}}}
    \newcommand{\VariableTok}[1]{\textcolor[rgb]{0.10,0.09,0.49}{{#1}}}
    \newcommand{\ControlFlowTok}[1]{\textcolor[rgb]{0.00,0.44,0.13}{\textbf{{#1}}}}
    \newcommand{\OperatorTok}[1]{\textcolor[rgb]{0.40,0.40,0.40}{{#1}}}
    \newcommand{\BuiltInTok}[1]{{#1}}
    \newcommand{\ExtensionTok}[1]{{#1}}
    \newcommand{\PreprocessorTok}[1]{\textcolor[rgb]{0.74,0.48,0.00}{{#1}}}
    \newcommand{\AttributeTok}[1]{\textcolor[rgb]{0.49,0.56,0.16}{{#1}}}
    \newcommand{\InformationTok}[1]{\textcolor[rgb]{0.38,0.63,0.69}{\textbf{\textit{{#1}}}}}
    \newcommand{\WarningTok}[1]{\textcolor[rgb]{0.38,0.63,0.69}{\textbf{\textit{{#1}}}}}
    
    
    % Define a nice break command that doesn't care if a line doesn't already
    % exist.
    \def\br{\hspace*{\fill} \\* }
    % Math Jax compatability definitions
    \def\gt{>}
    \def\lt{<}
    % Document parameters
    \title{true\_discontinuity\_depths\_velocity\_anomalies}
    
    
    

    % Pygments definitions
    
\makeatletter
\def\PY@reset{\let\PY@it=\relax \let\PY@bf=\relax%
    \let\PY@ul=\relax \let\PY@tc=\relax%
    \let\PY@bc=\relax \let\PY@ff=\relax}
\def\PY@tok#1{\csname PY@tok@#1\endcsname}
\def\PY@toks#1+{\ifx\relax#1\empty\else%
    \PY@tok{#1}\expandafter\PY@toks\fi}
\def\PY@do#1{\PY@bc{\PY@tc{\PY@ul{%
    \PY@it{\PY@bf{\PY@ff{#1}}}}}}}
\def\PY#1#2{\PY@reset\PY@toks#1+\relax+\PY@do{#2}}

\expandafter\def\csname PY@tok@w\endcsname{\def\PY@tc##1{\textcolor[rgb]{0.73,0.73,0.73}{##1}}}
\expandafter\def\csname PY@tok@c\endcsname{\let\PY@it=\textit\def\PY@tc##1{\textcolor[rgb]{0.25,0.50,0.50}{##1}}}
\expandafter\def\csname PY@tok@cp\endcsname{\def\PY@tc##1{\textcolor[rgb]{0.74,0.48,0.00}{##1}}}
\expandafter\def\csname PY@tok@k\endcsname{\let\PY@bf=\textbf\def\PY@tc##1{\textcolor[rgb]{0.00,0.50,0.00}{##1}}}
\expandafter\def\csname PY@tok@kp\endcsname{\def\PY@tc##1{\textcolor[rgb]{0.00,0.50,0.00}{##1}}}
\expandafter\def\csname PY@tok@kt\endcsname{\def\PY@tc##1{\textcolor[rgb]{0.69,0.00,0.25}{##1}}}
\expandafter\def\csname PY@tok@o\endcsname{\def\PY@tc##1{\textcolor[rgb]{0.40,0.40,0.40}{##1}}}
\expandafter\def\csname PY@tok@ow\endcsname{\let\PY@bf=\textbf\def\PY@tc##1{\textcolor[rgb]{0.67,0.13,1.00}{##1}}}
\expandafter\def\csname PY@tok@nb\endcsname{\def\PY@tc##1{\textcolor[rgb]{0.00,0.50,0.00}{##1}}}
\expandafter\def\csname PY@tok@nf\endcsname{\def\PY@tc##1{\textcolor[rgb]{0.00,0.00,1.00}{##1}}}
\expandafter\def\csname PY@tok@nc\endcsname{\let\PY@bf=\textbf\def\PY@tc##1{\textcolor[rgb]{0.00,0.00,1.00}{##1}}}
\expandafter\def\csname PY@tok@nn\endcsname{\let\PY@bf=\textbf\def\PY@tc##1{\textcolor[rgb]{0.00,0.00,1.00}{##1}}}
\expandafter\def\csname PY@tok@ne\endcsname{\let\PY@bf=\textbf\def\PY@tc##1{\textcolor[rgb]{0.82,0.25,0.23}{##1}}}
\expandafter\def\csname PY@tok@nv\endcsname{\def\PY@tc##1{\textcolor[rgb]{0.10,0.09,0.49}{##1}}}
\expandafter\def\csname PY@tok@no\endcsname{\def\PY@tc##1{\textcolor[rgb]{0.53,0.00,0.00}{##1}}}
\expandafter\def\csname PY@tok@nl\endcsname{\def\PY@tc##1{\textcolor[rgb]{0.63,0.63,0.00}{##1}}}
\expandafter\def\csname PY@tok@ni\endcsname{\let\PY@bf=\textbf\def\PY@tc##1{\textcolor[rgb]{0.60,0.60,0.60}{##1}}}
\expandafter\def\csname PY@tok@na\endcsname{\def\PY@tc##1{\textcolor[rgb]{0.49,0.56,0.16}{##1}}}
\expandafter\def\csname PY@tok@nt\endcsname{\let\PY@bf=\textbf\def\PY@tc##1{\textcolor[rgb]{0.00,0.50,0.00}{##1}}}
\expandafter\def\csname PY@tok@nd\endcsname{\def\PY@tc##1{\textcolor[rgb]{0.67,0.13,1.00}{##1}}}
\expandafter\def\csname PY@tok@s\endcsname{\def\PY@tc##1{\textcolor[rgb]{0.73,0.13,0.13}{##1}}}
\expandafter\def\csname PY@tok@sd\endcsname{\let\PY@it=\textit\def\PY@tc##1{\textcolor[rgb]{0.73,0.13,0.13}{##1}}}
\expandafter\def\csname PY@tok@si\endcsname{\let\PY@bf=\textbf\def\PY@tc##1{\textcolor[rgb]{0.73,0.40,0.53}{##1}}}
\expandafter\def\csname PY@tok@se\endcsname{\let\PY@bf=\textbf\def\PY@tc##1{\textcolor[rgb]{0.73,0.40,0.13}{##1}}}
\expandafter\def\csname PY@tok@sr\endcsname{\def\PY@tc##1{\textcolor[rgb]{0.73,0.40,0.53}{##1}}}
\expandafter\def\csname PY@tok@ss\endcsname{\def\PY@tc##1{\textcolor[rgb]{0.10,0.09,0.49}{##1}}}
\expandafter\def\csname PY@tok@sx\endcsname{\def\PY@tc##1{\textcolor[rgb]{0.00,0.50,0.00}{##1}}}
\expandafter\def\csname PY@tok@m\endcsname{\def\PY@tc##1{\textcolor[rgb]{0.40,0.40,0.40}{##1}}}
\expandafter\def\csname PY@tok@gh\endcsname{\let\PY@bf=\textbf\def\PY@tc##1{\textcolor[rgb]{0.00,0.00,0.50}{##1}}}
\expandafter\def\csname PY@tok@gu\endcsname{\let\PY@bf=\textbf\def\PY@tc##1{\textcolor[rgb]{0.50,0.00,0.50}{##1}}}
\expandafter\def\csname PY@tok@gd\endcsname{\def\PY@tc##1{\textcolor[rgb]{0.63,0.00,0.00}{##1}}}
\expandafter\def\csname PY@tok@gi\endcsname{\def\PY@tc##1{\textcolor[rgb]{0.00,0.63,0.00}{##1}}}
\expandafter\def\csname PY@tok@gr\endcsname{\def\PY@tc##1{\textcolor[rgb]{1.00,0.00,0.00}{##1}}}
\expandafter\def\csname PY@tok@ge\endcsname{\let\PY@it=\textit}
\expandafter\def\csname PY@tok@gs\endcsname{\let\PY@bf=\textbf}
\expandafter\def\csname PY@tok@gp\endcsname{\let\PY@bf=\textbf\def\PY@tc##1{\textcolor[rgb]{0.00,0.00,0.50}{##1}}}
\expandafter\def\csname PY@tok@go\endcsname{\def\PY@tc##1{\textcolor[rgb]{0.53,0.53,0.53}{##1}}}
\expandafter\def\csname PY@tok@gt\endcsname{\def\PY@tc##1{\textcolor[rgb]{0.00,0.27,0.87}{##1}}}
\expandafter\def\csname PY@tok@err\endcsname{\def\PY@bc##1{\setlength{\fboxsep}{0pt}\fcolorbox[rgb]{1.00,0.00,0.00}{1,1,1}{\strut ##1}}}
\expandafter\def\csname PY@tok@kc\endcsname{\let\PY@bf=\textbf\def\PY@tc##1{\textcolor[rgb]{0.00,0.50,0.00}{##1}}}
\expandafter\def\csname PY@tok@kd\endcsname{\let\PY@bf=\textbf\def\PY@tc##1{\textcolor[rgb]{0.00,0.50,0.00}{##1}}}
\expandafter\def\csname PY@tok@kn\endcsname{\let\PY@bf=\textbf\def\PY@tc##1{\textcolor[rgb]{0.00,0.50,0.00}{##1}}}
\expandafter\def\csname PY@tok@kr\endcsname{\let\PY@bf=\textbf\def\PY@tc##1{\textcolor[rgb]{0.00,0.50,0.00}{##1}}}
\expandafter\def\csname PY@tok@bp\endcsname{\def\PY@tc##1{\textcolor[rgb]{0.00,0.50,0.00}{##1}}}
\expandafter\def\csname PY@tok@fm\endcsname{\def\PY@tc##1{\textcolor[rgb]{0.00,0.00,1.00}{##1}}}
\expandafter\def\csname PY@tok@vc\endcsname{\def\PY@tc##1{\textcolor[rgb]{0.10,0.09,0.49}{##1}}}
\expandafter\def\csname PY@tok@vg\endcsname{\def\PY@tc##1{\textcolor[rgb]{0.10,0.09,0.49}{##1}}}
\expandafter\def\csname PY@tok@vi\endcsname{\def\PY@tc##1{\textcolor[rgb]{0.10,0.09,0.49}{##1}}}
\expandafter\def\csname PY@tok@vm\endcsname{\def\PY@tc##1{\textcolor[rgb]{0.10,0.09,0.49}{##1}}}
\expandafter\def\csname PY@tok@sa\endcsname{\def\PY@tc##1{\textcolor[rgb]{0.73,0.13,0.13}{##1}}}
\expandafter\def\csname PY@tok@sb\endcsname{\def\PY@tc##1{\textcolor[rgb]{0.73,0.13,0.13}{##1}}}
\expandafter\def\csname PY@tok@sc\endcsname{\def\PY@tc##1{\textcolor[rgb]{0.73,0.13,0.13}{##1}}}
\expandafter\def\csname PY@tok@dl\endcsname{\def\PY@tc##1{\textcolor[rgb]{0.73,0.13,0.13}{##1}}}
\expandafter\def\csname PY@tok@s2\endcsname{\def\PY@tc##1{\textcolor[rgb]{0.73,0.13,0.13}{##1}}}
\expandafter\def\csname PY@tok@sh\endcsname{\def\PY@tc##1{\textcolor[rgb]{0.73,0.13,0.13}{##1}}}
\expandafter\def\csname PY@tok@s1\endcsname{\def\PY@tc##1{\textcolor[rgb]{0.73,0.13,0.13}{##1}}}
\expandafter\def\csname PY@tok@mb\endcsname{\def\PY@tc##1{\textcolor[rgb]{0.40,0.40,0.40}{##1}}}
\expandafter\def\csname PY@tok@mf\endcsname{\def\PY@tc##1{\textcolor[rgb]{0.40,0.40,0.40}{##1}}}
\expandafter\def\csname PY@tok@mh\endcsname{\def\PY@tc##1{\textcolor[rgb]{0.40,0.40,0.40}{##1}}}
\expandafter\def\csname PY@tok@mi\endcsname{\def\PY@tc##1{\textcolor[rgb]{0.40,0.40,0.40}{##1}}}
\expandafter\def\csname PY@tok@il\endcsname{\def\PY@tc##1{\textcolor[rgb]{0.40,0.40,0.40}{##1}}}
\expandafter\def\csname PY@tok@mo\endcsname{\def\PY@tc##1{\textcolor[rgb]{0.40,0.40,0.40}{##1}}}
\expandafter\def\csname PY@tok@ch\endcsname{\let\PY@it=\textit\def\PY@tc##1{\textcolor[rgb]{0.25,0.50,0.50}{##1}}}
\expandafter\def\csname PY@tok@cm\endcsname{\let\PY@it=\textit\def\PY@tc##1{\textcolor[rgb]{0.25,0.50,0.50}{##1}}}
\expandafter\def\csname PY@tok@cpf\endcsname{\let\PY@it=\textit\def\PY@tc##1{\textcolor[rgb]{0.25,0.50,0.50}{##1}}}
\expandafter\def\csname PY@tok@c1\endcsname{\let\PY@it=\textit\def\PY@tc##1{\textcolor[rgb]{0.25,0.50,0.50}{##1}}}
\expandafter\def\csname PY@tok@cs\endcsname{\let\PY@it=\textit\def\PY@tc##1{\textcolor[rgb]{0.25,0.50,0.50}{##1}}}

\def\PYZbs{\char`\\}
\def\PYZus{\char`\_}
\def\PYZob{\char`\{}
\def\PYZcb{\char`\}}
\def\PYZca{\char`\^}
\def\PYZam{\char`\&}
\def\PYZlt{\char`\<}
\def\PYZgt{\char`\>}
\def\PYZsh{\char`\#}
\def\PYZpc{\char`\%}
\def\PYZdl{\char`\$}
\def\PYZhy{\char`\-}
\def\PYZsq{\char`\'}
\def\PYZdq{\char`\"}
\def\PYZti{\char`\~}
% for compatibility with earlier versions
\def\PYZat{@}
\def\PYZlb{[}
\def\PYZrb{]}
\makeatother


    % Exact colors from NB
    \definecolor{incolor}{rgb}{0.0, 0.0, 0.5}
    \definecolor{outcolor}{rgb}{0.545, 0.0, 0.0}



    
    % Prevent overflowing lines due to hard-to-break entities
    \sloppy 
    % Setup hyperref package
    \hypersetup{
      breaklinks=true,  % so long urls are correctly broken across lines
      colorlinks=true,
      urlcolor=urlcolor,
      linkcolor=linkcolor,
      citecolor=citecolor,
      }
    % Slightly bigger margins than the latex defaults
    
    \geometry{verbose,tmargin=1in,bmargin=1in,lmargin=1in,rmargin=1in}
    
    

    \begin{document}
    
    
    \maketitle
    
    

    
    \section{Determination of the true discontinuity depths and velocity
anomalies above a discontinuity using apparent
depths}\label{determination-of-the-true-discontinuity-depths-and-velocity-anomalies-above-a-discontinuity-using-apparent-depths}

    \subsubsection{Imaging mantle discontinuities using multiply-reflected
P-to-S
conversions}\label{imaging-mantle-discontinuities-using-multiply-reflected-p-to-s-conversions}

Stephen S. Gao, Kelly H. Liu

    After the moveout corrections, the arrival times and the corresponding
discontinuity depths can be related using depth--time--velocity
relations for vertically incident waves. Under the assumption of a
constant velocity structure beneath the area sampled by the \(Pds\) and
\(Ppds\) phases, the arrival time difference between the converted
S-wave and the direct P-wave can be expressed as:

    \[
T_{s - p}^{(Pds)} = \frac{H_{T}}{V_{s0} + \delta V_{s}} - \frac{H_{T}}{V_{p0} + \delta V_{p}}(1)
\]

    and that for \(Ppds\) is:

    \[
T_{s - p}^{(Ppds)} = \frac{H_{T}}{V_{s0} + \delta V_{s}} + \frac{H_{T}}{V_{p0} + \delta V_{p}} (2)
\]

    where:

\(H_{T}\) is the true depth of the discontinuity; \(V_{p0}\) and
\(V_{s0}\) are the mean P- and S-wave velocities in the IASP91 standard
Earth model above the discontinuity; \(\delta V_{p}\) and
\(\delta V_{w}\) are the P- and S-wave velocity anomalies. The use of
the velocities in the standard Earth model leads to an apparent depth of
\(H_{A}\) , and the corresponding travel-times are the same as those in
Eqs. (1) and (2) and can be expressed as:

    \[
T_{s - p}^{(Pds)} = \frac{H_{A}^{(Pds)}}{V_{s0}}  -  \frac{H_{A}^{(Pds)}}{V_{p0}} (3)
\]

    and:

    \[
T_{s - p}^{(Ppds)} = \frac{H_{A}^{(Ppds)}}{V_{s0}}  +  \frac{H_{A}^{(Ppds)}}{V_{p0}} (4)
\]

    Equating Eqs. (1) and (3) to find the \(H_{A}^{(Pds)}\), we get:

    \[
 \frac{H_{T}}{V_{s0} + \delta V_{s}} - \frac{H_{T}}{V_{p0} + \delta V_{p}} = \frac{H_{A}^{(Pds)}}{V_{s0}}  -  \frac{H_{A}^{(Pds)}}{V_{p0}}
\]

    \begin{itemize}
\tightlist
\item
  isolating \(H_{A}^{(Pds)}\) and \(H_{T}\):
\end{itemize}

    \[
 (\frac{1}{V_{s0} + \delta V_{s}} - \frac{1}{V_{p0} + \delta V_{p}}) .  H_{T} = (\frac{1}{V_{s0}}  -  \frac{1}{V_{p0}}) . H_{A}^{(Pds)}
\]

    \begin{itemize}
\tightlist
\item
  isolating \(H_{A}^{(Pds)}\):
\end{itemize}

    \[
 \frac{(V_{p0} + \delta V_{p}) - (V_{s0} + \delta V_{s})}{(V_{s0} + \delta V_{s}) . (V_{p0} + \delta V_{p})} . H_{T} = \frac{V_{p0}-V_{s0}}{V_{p0} . V_{s0}} .  H_{A}^{(Pds)}
\] \[
 \frac{V_{p0} + \delta V_{p} - V_{s0} - \delta V_{s}}{(V_{s0} + \delta V_{s}) . (V_{p0} + \delta V_{p})} . H_{T} = \frac{V_{p0}-V_{s0}}{V_{p0} . V_{s0}} .  H_{A}^{(Pds)}
\] \[
H_{A}^{(Pds)} = \frac{V_{p0} . V_{s0}}{V_{p0}-V_{s0}} . \frac{V_{p0} + \delta V_{p} - V_{s0} - \delta V_{s}}{(V_{s0} + \delta V_{s}) . (V_{p0} + \delta V_{p})} . H_{T} (5)
\] 

    Equating Eqs. (2) and (4) to find the \(H_{A}^{(Ppds)}\), we have:

    \[
\frac{H_{T}}{V_{s0} + \delta V_{s}} + \frac{H_{T}}{V_{p0} + \delta V_{p}} = \frac{H_{A}^{(Ppds)}}{V_{s0}}  +  \frac{H_{A}^{(Ppds)}}{V_{p0}}
\]

    \begin{itemize}
\tightlist
\item
  isolating \(H_{A}^{(Ppds)}\) and \(H_{T}\):
\end{itemize}

    \[
 (\frac{1}{V_{s0} + \delta V_{s}} + \frac{1}{V_{p0} + \delta V_{p}}) .  H_{T} = (\frac{1}{V_{s0}}  +  \frac{1}{V_{p0}}) . H_{A}^{(Ppds)}
\]

    \begin{itemize}
\tightlist
\item
  isolating \(H_{A}^{(Ppds)}\):
\end{itemize}

    \[
 \frac{(V_{p0} + \delta V_{p}) + (V_{s0} + \delta V_{s})}{(V_{s0} + \delta V_{s}) . (V_{p0} + \delta V_{p})} . H_{T} = \frac{V_{p0}-V_{s0}}{V_{p0} . V_{s0}} .  H_{A}^{(Ppds)}
\] \[
 \frac{V_{p0} + \delta V_{p} + V_{s0} + \delta V_{s}}{(V_{s0} + \delta V_{s}) . (V_{p0} + \delta V_{p})} . H_{T} = \frac{V_{p0} + V_{s0}}{V_{p0} . V_{s0}} .  H_{A}^{(Ppds)}
\] \[
H_{A}^{(Ppds)} = \frac{V_{p0} . V_{s0}}{V_{p0} + V_{s0}} . \frac{V_{p0} + \delta V_{p} + V_{s0} + \delta V_{s}}{(V_{s0} + \delta V_{s}) . (V_{p0} + \delta V_{p})} . H_{T} (6)
\] 

    Taking the ratio of the equations (5) e (6) results in:

    \[
\frac{H_{A}^{(Pds)}}{H_{A}^{(Ppds)}}
\] 

    \[
\frac{H_{A}^{(Pds)}}{H_{A}^{(Ppds)}} = \frac{\frac{V_{p0} . V_{s0}}{V_{p0}-V_{s0}} . \frac{V_{p0} + \delta V_{p} - V_{s0} - \delta V_{s}}{(V_{s0} + \delta V_{s}) . (V_{p0} + \delta V_{p})} . H_{T}}{\frac{V_{p0} . V_{s0}}{V_{p0} + V_{s0}} . \frac{V_{p0} + \delta V_{p} + V_{s0} + \delta V_{s}}{(V_{s0} + \delta V_{s}) . (V_{p0} + \delta V_{p})} . H_{T}}
\]

    \begin{itemize}
\tightlist
\item
  taking \(\alpha\) as \(V_{p0} − V_{s0}\) , and \(\beta\) as
  \(V_{p0} + V_{s0}\), we have:
\end{itemize}

    \[
\frac{H_{A}^{(Pds)}}{H_{A}^{(Ppds)}} = \frac{\frac{V_{p0} . V_{s0}}{\alpha} . \frac{\alpha + \delta V_{p} - \delta V_{s}}{(V_{s0} + \delta V_{s}) . (V_{p0} + \delta V_{p})} . H_{T}}{\frac{V_{p0} . V_{s0}}{\beta} . \frac{\beta + \delta V_{p} + \delta V_{s}}{(V_{s0} + \delta V_{s}) . (V_{p0} + \delta V_{p})} . H_{T}}
\] 

    \[
\frac{H_{A}^{(Pds)}}{H_{A}^{(Ppds)}} = \frac{\frac{(V_{p0} . V_{s0}) . (\alpha + \delta V_{p} - \delta V_{s})}{\alpha . (V_{s0} + \delta V_{s}) . (V_{p0} + \delta V_{p})} . H_{T}}{\frac{(V_{p0} . V_{s0}). (\beta + \delta V_{p} + \delta V_{s})}{\beta . (V_{s0} + \delta V_{s}) . (V_{p0} + \delta V_{p})} . H_{T}}
\]

    \[
\frac{H_{A}^{(Pds)}}{H_{A}^{(Ppds)}} = \frac{(V_{p0} . V_{s0}) . (\alpha + \delta V_{p} - \delta V_{s})}{\alpha . (V_{s0} + \delta V_{s}) . (V_{p0} + \delta V_{p})} . H_{T} . (\frac{\beta . (V_{s0} + \delta V_{s}) . (V_{p0} + \delta V_{p})}{(V_{p0} . V_{s0}). (\beta + \delta V_{p} + \delta V_{s})} . \frac{1}{H_{T}})
\]

    \[
\frac{H_{A}^{(Pds)}}{H_{A}^{(Ppds)}} = \frac{\alpha + \delta V_{p} - \delta V_{s}}{\alpha}. \frac{\beta}{\beta + \delta V_{p} + \delta V_{s}}
\]

    \begin{itemize}
\tightlist
\item
  reorganizing some terms, we get:
\end{itemize}

    \[
\frac{H_{A}^{(Pds)}}{H_{A}^{(Ppds)}} = \frac{(\alpha + \delta V_{p} - \delta V_{s})}{(\beta + \delta V_{p} + \delta V_{s})}. \frac{\beta}{\alpha} (7)
\]

    Eq. (7) indicates that \(H_{A}^{(Pds)}=H_{A}^{(Ppds)}=H_{T}\) when the
real velocities are the same as those in the standard Earth model
(i.e.,\(\delta V_{p} = \delta V_{s} = 0\)). The apparent depths obtained
from \(Pds\) and \(Ppds\) are different when the real velocities are
different from those in the standard Earth model. Thus the discrepancies
between the apparent depths using \(Pds\) and \(Ppds\) are diagnostics
of velocity anomalies above the discontinuities.

Under the assumption that the fractional P- and S-wave velocity
anomalies are proportional, i.e.,

    \[
\frac{\delta V_{s}}{V_{s0}}  = \gamma  \frac{\delta V_{p}}{V_{p0}} (8)
\]

    where \(\gamma\) is a constant, we can solve for \(\delta V_{p}\) using
Eqs. (7) and (8), i.e.,

    \begin{itemize}
\tightlist
\item
  isolating \(\delta V_{s}\) in equation (8):
\end{itemize}

    \[
\frac{\delta V_{s}}{V_{s0}}  = \gamma . \frac{\delta V_{p}}{V_{p0}}
\] \[
\delta V_{s} = \gamma . \frac{\delta V_{p}.V_{s0}}{V_{p0}}
\] \[
\delta V_{s} = \delta V_{p} . \frac{\gamma.V_{s0}}{V_{p0}}(9)
\]

    \begin{itemize}
\tightlist
\item
  developing the eq.(7):
\end{itemize}

    \[
\frac{H_{A}^{(Pds)}}{H_{A}^{(Ppds)}} = \frac{(\alpha + \delta V_{p} - \delta V_{s})}{(\beta + \delta V_{p} + \delta V_{s})}. \frac{\beta}{\alpha} (7)
\]

    \[
\frac{(H_{A}^{(Pds)} . \alpha)}{(H_{A}^{(Ppds)} . \beta}) = \frac{(\alpha + \delta V_{p} - \delta V_{s})}{(\beta + \delta V_{p} + \delta V_{s})}
\]

    \[
(H_{A}^{(Pds)} . \alpha . \beta) + (H_{A}^{(Pds)} . \alpha . \delta V_{p}) + (H_{A}^{(Pds)} . \alpha . \delta V_{s}) = (H_{A}^{(Ppds)} . \beta . \alpha) + (H_{A}^{(Ppds)} . \beta . \delta V_{p}) - (H_{A}^{(Ppds)} . \beta . \delta V_{s}) (10)
\]

    \begin{itemize}
\tightlist
\item
  isolating \(\delta V_{p}\) and \(\delta V_{s}\) in equation (10):
\end{itemize}

    \[
(H_{A}^{(Pds)} . \alpha . \beta) + (H_{A}^{(Pds)} . \alpha . \delta V_{p}) + (H_{A}^{(Pds)} . \alpha . \delta V_{s}) = (H_{A}^{(Ppds)} . \beta . \alpha) + (H_{A}^{(Ppds)} . \beta . \delta V_{p}) - (H_{A}^{(Ppds)} . \beta . \delta V_{s})
\]

    \[
(H_{A}^{(Pds)} . \alpha . \delta V_{p}) + (H_{A}^{(Pds)} . \alpha . \delta V_{s}) - (H_{A}^{(Ppds)} . \beta . \delta V_{p}) + (H_{A}^{(Ppds)} . \beta . \delta V_{s}) = (H_{A}^{(Ppds)} . \beta . \alpha) - (H_{A}^{(Pds)} . \alpha . \beta) (11)
\]

    \begin{itemize}
\tightlist
\item
  replacing eq. (9) in eq. (11):
\end{itemize}

    \[
(H_{A}^{(Pds)} . \alpha . \delta V_{p}) + (H_{A}^{(Pds)} . \alpha . (\delta V_{p} . \frac{\gamma.V_{s0}}{V_{p0}})) - (H_{A}^{(Ppds)} . \beta . \delta V_{p}) + (H_{A}^{(Ppds)} . \beta . (\delta V_{p} . \frac{\gamma.V_{s0}}{V_{p0}})) = \alpha . \beta . (H_{A}^{(Ppds)} - H_{A}^{(Pds)})
\]

    \[
(\alpha . \delta V_{p} . H_{A}^{(Pds)}) + (\alpha . \delta V_{p} . H_{A}^{(Pds)} . \frac{\gamma.V_{s0}}{V_{p0}}) - (\beta . \delta V_{p} . H_{A}^{(Ppds)}) + (\beta . \delta V_{p} .  H_{A}^{(Ppds)} . \frac{\gamma.V_{s0}}{V_{p0}}) = \alpha . \beta . (H_{A}^{(Ppds)} - H_{A}^{(Pds)})
\]

    \begin{itemize}
\tightlist
\item
  isolating \(\delta V_{p}\)
\end{itemize}

    \[
\delta V_{p} . (\alpha . H_{A}^{(Pds)} + \alpha . H_{A}^{(Pds)} . \frac{\gamma.V_{s0}}{V_{p0}}) - \delta V_{p} . (\beta . H_{A}^{(Ppds)} - \beta .  H_{A}^{(Ppds)} . \frac{\gamma.V_{s0}}{V_{p0}}) = \alpha . \beta . (H_{A}^{(Ppds)} - H_{A}^{(Pds)})
\]

    \[
\delta V_{p} . ((\alpha . H_{A}^{(Pds)} + \alpha . H_{A}^{(Pds)} . \frac{\gamma.V_{s0}}{V_{p0}}) - (\beta . H_{A}^{(Ppds)} - \beta .  H_{A}^{(Ppds)} . \frac{\gamma.V_{s0}}{V_{p0}})) = \alpha . \beta . (H_{A}^{(Ppds)} - H_{A}^{(Pds)})
\]

    \[
\delta V_{p} = \frac{\alpha . \beta . (H_{A}^{(Ppds)} - (H_{A}^{(Pds)})}{(\alpha . H_{A}^{(Pds)} + \alpha . H_{A}^{(Pds)} . \frac{\gamma.V_{s0}}{V_{p0}}) - (\beta . H_{A}^{(Ppds)} - \beta .  H_{A}^{(Ppds)} . \frac{\gamma.V_{s0}}{V_{p0}})}
\] 

    \[
\delta V_{p} = \frac{\alpha . \beta . (H_{A}^{(Ppds)} - (H_{A}^{(Pds)})}{\alpha . (1 + \frac{\gamma.V_{s0}}{V_{p0}}) . H_{A}^{(Pds)} - \beta . (1 - \frac{\gamma.V_{s0}}{V_{p0}}). H_{A}^{(Ppds)}} (12)
\] 

    Once \(\delta V_{p}\) and \(\delta V_{s}\) are determined using Eqs.
(12) and (9), respectively, the true depth of the discontinuity,
\(H_{T}\) , can then be found using Eqs. (5) or (6).

    \[
H_{A}^{(Pds)} = \frac{V_{p0} . V_{s0}}{V_{p0}-V_{s0}} . \frac{V_{p0} + \delta V_{p} - V_{s0} - \delta V_{s}}{(V_{s0} + \delta V_{s}) . (V_{p0} + \delta V_{p})} . H_{T} (5)
\]

    \[
H_{A}^{(Ppds)} = \frac{V_{p0} . V_{s0}}{V_{p0} + V_{s0}} . \frac{V_{p0} + \delta V_{p} + V_{s0} + \delta V_{s}}{(V_{s0} + \delta V_{s}) . (V_{p0} + \delta V_{p})} . H_{T} (6)
\]

    \begin{itemize}
\tightlist
\item
  isolating \(H_{T}\) in the eq. (5):
\end{itemize}

    \[
H_{A}^{(Pds)} . \frac{V_{p0}-V_{s0}}{V_{p0} . V_{s0}}  = \frac{V_{p0} + \delta V_{p} - V_{s0} - \delta V_{s}}{(V_{s0} + \delta V_{s}) . (V_{p0} + \delta V_{p})} . H_{T}
\]

    \[
 H_{T} = H_{A}^{(Pds)} . \frac{V_{p0}-V_{s0}}{V_{p0} . V_{s0}} . \frac{(V_{s0} + \delta V_{s}) . (V_{p0} + \delta V_{p})}{V_{p0} + \delta V_{p} - V_{s0} - \delta V_{s}} (13)
\] 

    \begin{itemize}
\tightlist
\item
  isolating \(H_{T}\) in the eq. (6):
\end{itemize}

    \[
H_{A}^{(Ppds)} . \frac{V_{p0} + V_{s0}}{V_{p0} . V_{s0}} = \frac{V_{p0} + \delta V_{p} + V_{s0} + \delta V_{s}}{(V_{s0} + \delta V_{s}) . (V_{p0} + \delta V_{p})} . H_{T}
\]

    \[
H_{T} = H_{A}^{(Ppds)} . \frac{V_{p0} + V_{s0}}{V_{p0} . V_{s0}} . \frac{(V_{s0} + \delta V_{s}) . (V_{p0} + \delta V_{p})}{V_{p0} + \delta V_{p} + V_{s0} + \delta V_{s}}(14)
\] 

    From the resulting true depths of the d410 (\(H_{T}^{(4)}\)) and d660
(\(H_{T}^{(6)}\)) and the mean velocities above them
(\(V_{p}^{(4)}\),\(V_{p}^{(6)}\),\(V_{s}^{(4)}\), and \(V_{s}^{(6)}\)),
the velocity anomalies of the MTZ (\(\delta V_{p}^{(M)}\) and
\(\delta V_{s}^{(M)}\)) be found using Eq. (8):

    \[
\frac{\delta V_{s}^{(M)}}{V_{s0}^{(M)}}  = \gamma  \frac{\delta V_{p}^{(M)}}{V_{p0}^{(M)}} (15)
\]

    and the following partitioning relationship of the S- and P-wave
differential times, i.e

    \[
\frac{H_{T}^{(6)}}{V_{s}^{(6)}} - \frac{H_{T}^{(6)}}{V_{p}^{(6)}} - \frac{H_{T}^{(4)}}{V_{s}^{(4)}} + \frac{H_{T}^{(4)}}{V_{p}^{(4)}} = \frac{H_{T}^{(6)} - H_{T}^{(4)}}{V_{s0}^{(M)} + \delta V_{s}^{(M)}} - \frac{H_{T}^{(6)} - H_{T}^{(4)}}{V_{p0}^{(M)} + \delta V_{p}^{(M)}} (16)
\]


    % Add a bibliography block to the postdoc
    
    
    
    \end{document}
